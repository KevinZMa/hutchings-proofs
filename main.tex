\documentclass{article}
\usepackage{amsthm}
\usepackage{amsmath}
\usepackage{amssymb}
\usepackage{lineno}
\usepackage{xcolor}
\usepackage{hyperref}

% Make line numbers smaller, lighter gray, and further to the left
\renewcommand\linenumberfont{\normalfont\tiny\color{gray}}
\setlength\linenumbersep{2em}

\linenumbers

\newenvironment{answer}{
\begin{quote}}{
\end{quote}}

%%%

\title{Exercises for Introduction to Mathematical Arguments}
\author{Kevin Ma}
\date{\today\\\small{Based on notes by M.
Hutchings\footnote{\url{https://math.berkeley.edu/~hutching/teach/proofs.pdf}}}}

\newcommand{\negation}[1]{\textbf{Negation:} #1}

\begin{document}

\maketitle

\setcounter{section}{1}

\section{How to prove things}

\begin{enumerate}
  \item Prove the following statements; what is the negation of each
    of these statements?
    \begin{enumerate}
      \item For every integer $x$, if $x$ is even, then for every
        integer $y$, $xy$ is even.

        \begin{proof}
          Since $x$ is even, choose an integer $w$ such that $x = 2w$.

          Then, $xy = 2wy$.

          Let $z=wy$; then $xy=2z$, so $xy$ is even.
        \end{proof}

        \negation{$x$ is even, and there is an integer $y$ where $xy$ is odd.}

      \item For every integer $x$ and for every integer $y$, if $x$
        is odd and $y$ is odd then $x + y$ is even.

        \begin{proof}
          Since $x$ is odd, choose an integer $v$ such that $x=2v+1$.

          Since $y$ is odd, choose an integer $w$ such that $y=2w+1$.

          Then, $x+y=2v+2w+2$.

          Let $z=v+w+1$; then $x+y=2z$, so $x+y$ is even.
        \end{proof}

        \negation{There is an integer $x$ and an integer $y$ such
        that $x$ is odd, $y$ is odd, and $x+y$ is odd.}

      \item For every integer $x$, if $x$ is odd then $x^3$ is odd.

        \begin{proof}
          Since $x$ is odd, choose an integer $w$ such that $x=2w+1$.

          Then, $x^3 = 8w^3+12w^2+6w+1$.

          Let $z=4w^3+6w^2+3w$; then $x^3=2z+1$, so $x^3$ is odd.
        \end{proof}

        \negation{There is an integer $x$ such that $x$ is odd and
        $x^3$ is even.}

    \end{enumerate}
  \item Prove that for every integer $x$, $x + 4$ is odd if and only
    if $x + 7$ is even.

    \begin{proof}
      ($\Rightarrow$) Suppose $x+4$ is odd.

      Choose an integer $v$ such that $x+4=2v+1$.

      Then, $x+7 = 2v+4$.

      Let $z=v+2$; then $x+7 = 2z$, so $x+7$ is even.

      ($\Leftarrow$) Suppose $x+7$ is even.

      Choose an integer $w$ such that $x+7=2w$.

      Then, $x+4=2w-3=2(w-2)+1$.

      Let $z=w-2$; then $x+4=2z+1$, so $x+4$ is odd.
    \end{proof}

  \item Figure out whether the statement we negated in \S1.3 is true
    or false, and prove it (or its negation).

    The statement from \S1.3 is:
    \begin{align}
      (\forall x \in \mathbb{Z}) \left((\exists y \in \mathbb{Z}) x =
      3y + 1\right) \Rightarrow \left((\exists y \in \mathbb{Z}) x^2
      = 3y + 1\right).
    \end{align}

    \begin{proof}
      Let $x$ be an integer such that there exists an integer $y$
      where $x=3y+1$.

      By squaring both sides,
      \begin{align}
        x^2 &= (3y+1)^2 \\
        &= 9y^2+6y+1 \\
        &= 3(3y^2+2y) + 1.
      \end{align}

      Let $z=3y^2+2y$.
      Since $z$ must be an integer, there exists an integer $z$ where
      $x^2=3z+1$.
    \end{proof}

  \item Prove that for every integer $x$, if $x$ is odd then there
    exists an integer $y$ such that $x^2 = 8y + 1$.

    \begin{proof}
      Since $x$ is odd, let $w$ be an integer such that $x=2w+1$.

      By squaring both sides,
      \begin{align}
        x^2 &= (2w+1)^2 \\
        &= 4w^2+4w+1 \\
        &= 4w(w+1)+1 \\
        &= 8(\frac{1}{2}w(w+1))+1
      \end{align}

      Since one of $w$ and $w+1$ must be even, their product $w(w+1)$
      must also be even.

      Therefore, there exists an integer $y$ where $w(w+1)=2y$,
      meaning $y=\frac{1}{2}w(w+1)$.

      Thus, by substitution, $x^2$ can be represented in the form
      $8y+1$, where $y$ is an integer.
    \end{proof}

\end{enumerate}

\section{More proof techniques}

\begin{enumerate}

  \item Prove that the inverse of a given element $x \in G$ is unique.

    \begin{proof}
      Let $e$ be the identity element and let $v,w$ be elements of
      $G$ satisfying $vx=e$ and $xw=e$, by the definition of an inverse.
      \begin{align}
        v &= ve & & \text{(definition of identity element)} \\
        &= v(xw) & & \text{(substitution)} \\
        &= (vx)w & & \text{(associative property)} \\
        &= ew & & \text{(substitution)} \\
        &= w & & \text{(definition of identity element)}
      \end{align}
      Thus, the inverse of a given element $x \in G$ is unique.
    \end{proof}

\end{enumerate}

\section{Proof by induction}

\begin{enumerate}

  \item Fix a real number $x \neq 1$. Show that for every positive integer $n$,
    \[
      1 + x + x^2 + \ldots + x^n = \frac{x^{n+1} - 1}{x - 1}.
    \]

    \begin{proof}
      \textit{Base case:} For $n=1$:
      \[
        1+x=\frac{x^2-1}{x-1}=x+1.
      \]

      \textit{Inductive step:} Add $x^{n+1}$ to both sides.
      \begin{align*}
        1+x+x^2+\ldots+x^n+x^{n+1}&=\frac{x^{n+1}-1}{x-1}+x^{n+1} \\
        &= \frac{x^{n+1}-1+x^{n+1}(x-1)}{x-1} \\
        &= \frac{x^{n+1}(1+x-1)-1}{x-1} \\
        &= \frac{x^{n+2}-1}{x-1}
        .
      \end{align*}
    \end{proof}

  \item Guess a formula for
    \[
      \frac{1}{1 \cdot 2} + \frac{1}{2 \cdot 3} + \ldots + \frac{1}{n(n+1)}
    \]
    and prove it by induction. \textit{Hint:} Compute the above
    expression for some small values of $n$.

    \begin{answer}
      For $n=1$: $S=\frac{1}{2}$.

      For $n=2$: $S=\frac{1}{2}+\frac{1}{6}=\frac{2}{3}$.

      For $n=3$: $S=\frac{4}{6}+\frac{1}{12}=\frac{3}{4}$.

      For $n=4$: $S=\frac{3}{4}+\frac{1}{20}=\frac{4}{5}$.

      The pattern looks like $\frac{n}{n+1}$.

      \begin{proof}
        \textit{Base case:}
        \begin{align*}
          \frac{1}{1 \cdot 2} = \frac{1}{1+1} = \frac{1}{2}
          .
        \end{align*}

        \textit{Inductive step:} To prove the sum with the $(n+1)$th
        term is $\frac{n+1}{n+2}$, add the next term
        $\frac{1}{(n+1)(n+2)}$ to both sides.

        \begin{align*}
          \frac{1}{1 \cdot 2} + \frac{1}{2 \cdot 3} &+ \ldots +
          \frac{1}{n(n+1)} + \frac{1}{(n+1)(n+2)}\\
          &= \frac{n}{n+1} + \frac{1}{(n+1)(n+2)} \\
          &= \frac{n(n+2)+1}{(n+1)(n+2)} \\
          &= \frac{n^2+2n+1}{(n+1)(n+2)} \\
          &= \frac{n+1}{n+2}
          .
        \end{align*}
      \end{proof}
    \end{answer}

  \item Show that a $2^n \times 2^n$ checkerboard with one square
    removed can be tiled with L-triominoes. (An \textbf{L-triomino}
    is a shape consisting of three squares joined in an `L'-shape.)

    \begin{proof}
      \textit{Base case:} When removing any square of a $2\times 2$
      checkerboard, 3 squares remain and must form an L-triomino.

      \textit{Inductive step:} A $2^{n+1}\times 2^{n+1}$ board can be
      divided into four $2^n \times 2^n$ boards by dividing it in
      half (by 2) both vertically and horizontally. We will refer to
      each of these boards as \textbf{quadrants}.

      A square is removed from one of the four quadrants. By our
      induction hypothesis, this quadrant (a $2^n \times 2^n$ board)
      can be tiled with L-triominoes.

      Because $n\ge 1$, the board will always have at least 4
      squares. Consider the center 4 squares. Since this is a
      $2\times 2$ ``board'' with one square from a quadrant that has
      already been accounted for or ``occupied,'' the base case
      applies to this region, and an L-triomino can be placed here.

      Now, each of the remaining quadrants has one square occupied,
      which has the same effect as being removed. By our induction
      hypothesis, the remaining quadrants can also be tiled by L-triominoes.
    \end{proof}

  \item Show that the smallest element of a nonempty set of positive
    integers is unique.

    \begin{proof}
      \textit{Base case:} Consider an $n$-element set of positive
      integers where $n=1$. The smallest (only) element is unique.

      \textit{Inductive step:} Our inductive hypothesis states that
      the smallest element in a set with $n$ positive integers is
      unique. Let $z$ be the smallest element of the set, excluding
      element $n+1$, which we will refer to as $m$.
    \end{proof}

\end{enumerate}

\appendix
\section{Sets}

\begin{enumerate}
  \item List separately the elements and the subsets of $\{\{1,
    \{2\}\}, \{3\}\}$. (There are 2 elements and 4 subsets.)
    \begin{answer}
      Elements: \{1, \{2\}\}, \{3\}

      Subsets: $\emptyset$, \{\{1, \{2\}\}\}, \{\{3\}\}, \{\{1, \{2\}\}, \{3\}\}
    \end{answer}

  \item Explain why if $A \subset B$ and $B \subset C$, then $A \subset C$.
  \begin{answer}
    Since every element of $A$ is in $B$, and every element of $B$ is in $C$, every element in $A$ is in $C$.    
  \end{answer}

  \item If a set has exactly $n$ elements, how many subsets does it have? Why?
  \begin{answer}
    Every element in the set has 2 options, to be included or excluded. Thus, there are $2^n$ possible subsets.
  \end{answer}

  \item We have repeatedly used the words, `\textit{the} empty set'. Is this
    justified? If $A$ and $B$ are both sets that contain no elements,
    then is $A$ necessarily equal to $B$?
  \begin{answer}
    Yes, because the elements---or lack thereof---of both $A$ and $B$ are identical.
  \end{answer}

  \item Which of the following statements are true, and which are false? Why?
    \begin{enumerate}
      \item $\{\emptyset\} \cup \emptyset = \{\emptyset\}$
      \item $\{\emptyset\} \cup \{\emptyset\} = \{\emptyset\}$
      \item $\{\emptyset\} \cap \{\{\emptyset\}\} = \emptyset$
      \item $\{\emptyset\} \cap \{\{\emptyset\}\} = \{\emptyset\}$
    \end{enumerate}

  \item Prove all of the properties of union, intersection, and set
    difference that we stated without proof in the text.

  \item Show that $A \cup (B-C) = (A \cup B)-(A \cap C)$. Is it
    always true that $A \cup (B-C) = (A \cup B) - (A \cup C)$?

  \item Find some more set theoretic identities and prove them.
\end{enumerate}

\end{document}
