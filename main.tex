\documentclass{article}
\usepackage{amsthm}
\usepackage{amsmath}
\usepackage{amssymb}
\usepackage{lineno}
\usepackage{xcolor}
\usepackage{hyperref}


% Make line numbers smaller, lighter gray, and further to the left
\renewcommand\linenumberfont{\normalfont\tiny\color{gray}}
\setlength\linenumbersep{2em}

\linenumbers

\title{Exercises for Introduction to Mathematical Arguments}
\author{Kevin Ma}
\date{\today\\\small{Based on notes by M. Hutchings\footnote{\href{https://math.berkeley.edu/\~{}hutching/teach/proofs.pdf}{https://math.berkeley.edu/$\sim$hutching/teach/proofs.pdf}}}}

\newcommand{\negation}[1]{\textbf{Negation:} #1}

\begin{document}

\maketitle

\setcounter{section}{1}

\section{How to prove things}

\begin{enumerate}
  \item Prove the following statements; what is the negation of each
    of these statements?
    \begin{enumerate}
      \item For every integer $x$, if $x$ is even, then for every
        integer $y$, $xy$ is even.

        \begin{proof}
          Since $x$ is even, choose an integer $w$ such that $x = 2w$.

          Then, $xy = 2wy$.

          Let $z=wy$; then $xy=2z$, so $xy$ is even.
        \end{proof}

        \negation{$x$ is even, and there is an integer $y$ where $xy$ is odd.}

      \item For every integer $x$ and for every integer $y$, if $x$
        is odd and $y$ is odd then $x + y$ is even.

        \begin{proof}
          Since $x$ is odd, choose an integer $v$ such that $x=2v+1$.
          
          Since $y$ is odd, choose an integer $w$ such that $y=2w+1$.

          Then, $x+y=2v+2w+2$.

          Let $z=v+w+1$; then $x+y=2z$, so $x+y$ is even.
        \end{proof}

        \negation{There is an integer $x$ and an integer $y$ such that $x$ is odd, $y$ is odd, and $x+y$ is odd.}

      \item For every integer $x$, if $x$ is odd then $x^3$ is odd.

        \begin{proof}
          Since $x$ is odd, choose an integer $w$ such that $x=2w+1$.

          Then, $x^3 = 8w^3+12w^2+6w+1$.

          Let $z=4w^3+6w^2+3w$; then $x^3=2z+1$, so $x^3$ is odd.
        \end{proof}

        \negation{There is an integer $x$ such that $x$ is odd and $x^3$ is even.}

    \end{enumerate}
  \item Prove that for every integer $x$, $x + 4$ is odd if and only
    if $x + 7$ is even.

    \begin{proof}
      ($\Rightarrow$) Suppose $x+4$ is odd.

      Choose an integer $v$ such that $x+4=2v+1$.

      Then, $x+7 = 2v+4$.

      Let $z=v+2$; then $x+7 = 2z$, so $x+7$ is even.

      ($\Leftarrow$) Suppose $x+7$ is even.

      Choose an integer $w$ such that $x+7=2w$.

      Then, $x+4=2w-3=2(w-2)+1$.

      Let $z=w-2$; then $x+4=2z+1$, so $x+4$ is odd.
    \end{proof}

  \item Figure out whether the statement we negated in \S1.3 is true
    or false, and prove it (or its negation).

    The statement from \S1.3 is:
    \begin{align}
      (\forall x \in \mathbb{Z}) \left((\exists y \in \mathbb{Z}) x = 3y + 1\right) \Rightarrow \left((\exists y \in \mathbb{Z}) x^2 = 3y + 1\right).
    \end{align}

    \begin{proof}
      Let $x$ be an integer such that there exists an integer $y$ where $x=3y+1$.

      By squaring both sides,
      \begin{align}
      x^2 &= (3y+1)^2 \\
      &= 9y^2+6y+1 \\
      &= 3(3y^2+2y) + 1.
      \end{align}
      
      Let $z=3y^2+2y$.
      Since $z$ must be an integer, there exists an integer $z$ where $x^2=3z+1$.
    \end{proof}

  \item Prove that for every integer $x$, if $x$ is odd then there
    exists an integer $y$ such that $x^2 = 8y + 1$.

    \begin{proof}
      Since $x$ is odd, let $w$ be an integer such that $x=2w+1$.

      By squaring both sides,
      \begin{align}
        x^2 &= (2w+1)^2 \\
        &= 4w^2+4w+1 \\
        &= 4w(w+1)+1 \\
        &= 8(\frac{1}{2}w(w+1))+1
      \end{align}

      Since one of $w$ and $w+1$ must be even, their product $w(w+1)$ must also be even.

      Therefore, there exists an integer $y$ where $w(w+1)=2y$, meaning $y=\frac{1}{2}w(w+1)$.
      
      Thus, by substitution, $x^2$ can be represented in the form $8y+1$, where $y$ is an integer.
    \end{proof}

\end{enumerate}

\end{document}
